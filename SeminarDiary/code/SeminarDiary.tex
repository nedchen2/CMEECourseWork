\documentclass[11pt]{article}

\usepackage{setspace}
\usepackage[colorlinks]{hyperref}
\usepackage{lineno}
\usepackage{booktabs}
\usepackage{graphicx}
\usepackage{float}
\usepackage{floatrow}
\usepackage{subfigure}
\usepackage{caption}
\usepackage{subcaption}
\usepackage{geometry}
\usepackage{multirow}
\usepackage{longtable}
\usepackage{lscape}
\usepackage{booktabs}


\title{\textbf{Seminar Diary}}

\author{Congjia Chen\footnotemark[1]}

\renewcommand{\thefootnote}{\fnsymbol{footnote}}
\footnotetext[1]{Department of Life Sciences (Silwood Park), Imperial College of Science, Technology and Medicine}

\date{2022}

\linespread{1.5}
\geometry{left=2cm,right=2cm,top=2cm,bottom=2cm}
\begin{document}
  \maketitle
  \newpage


\linenumbers
\section{Underground forests, fire, mammal browsers and the origin of African savanna}
Lecturer: Jonathan Davies , University of British Columbia
\newline
Date: 2021/10/14
\newline

Savanna is a mixed woodland-grassland ecosystem characterised by the trees being sufficiently widely spaced so that the canopy does not close. Given that they cover about 20\% of the world's vegetated land surface, Savannas are one of the world's major and important biomes. The transition from the tree to the suffrutescent life form has been interpreted as an adaptation to fire. Over large parts of their range, savannas occur in mosaics with forests leading to the idea that they are products of anthropogenic fire and deforestation. 

By combining the geographic information with the phylogeny extracted from DNA barcoding (It works by analysing a specific region of DNA. This region is called the DNA barcode. The sequence of this DNA barcode is then compared to a reference library which contains information of many species linked to their barcodes. Group the species together and get the idea of history and evolution), they can have the basic idea of how the biomes collape and "evolve" in the time dimension. Moreover, using geoxylic suffrutices (woody plants that develop their perennial stems and branches below ground, and produce above-ground stems annually ), White's underground forests of Africa, as markers for fire-maintained savanna, they find that the first-appeared time of these savannas occurred millions of years before humans began to fell and burn forests. As a support, using the marker of spiny trees evolution, they separate the evolution history between herbivore-adapted savanna with fire-maintained savanna.

The latitudinal expansion of savanna ecosystems and the retreat of forests likely played a key role in early human developments in Africa, and the current distribution of humans may still be closely tied to the distribution of woody flora on the continent.
    
\newpage
    
    
    
\linenumbers
\section{Birdwatching on a cosmic scale: The avian fossil record and the origin of modern bird diversity }
Lecturer: Daniel J.Feild, University of Cambridge.
\newline
Date: 2021/11/4
\newline

Living birds are the most diverse group of terrestrial vertebrate animals, comprising nearly 11,000 extant species. They inhabit virtually every corner of the modern world, and exhibit a mind-boggling variety of forms and lifestyles. But how has this awe-inspiring diversity arisen? This speech will explore recent research into how, where, and when the spectacular diversity of living birds, their specialised features, and their extraordinary phenotypic variety came about.

With the development of a wealth of new phenotypic and genomic data, people can get important new insights into these longstanding evolutionary questions. They hypothesis three key hypotheses: 
1) Deep evolutionary branch
- Only several deep lineages go cross the mass extinction  
2) Relatively small-bodied
- relatively small-bodied organisms may be more adaptive 
3) Non-tree-dwelling
- the fire and deforestation make the organisms non-tree-dwelling

Meanwhile, new bird fossils which is from about 66.8 million years ago (before the mass extinction) and advanced visualisation techniques (e.g.  CT and computer techniques) provide real evidence for the evolutionary hypotheses. By comparing the galloanserae with duck-like and chicken-like lineages, they provide evidence for the deep branch common ancestor between chicken-like birds and duck-like birds. The 3D print of the new fossil skull is relatively smaller than the other lineages which provide evidence for the second hypotheses. By looking at the Hindlimb index, they conclude that the newly discovered fossil should be ground-dwelling.  

Understanding the earliest stages of modern bird evolutionary history will give us a direct insight to the extinction of giant dinosaurs.The discovery of the world's oldest modern bird fossil has helped us understand the earliest stages of modern bird evolutionary history. This information will help us to unravel the effects of the mass extinction on avian ecology, anatomy, and diversity.

    
    
\newpage
  
    
    
    
\linenumbers
\section{The dynamics and impacts of marine human-wildlife conflict}
Lecturer: Dr Katrina J Davis , University of Oxford
\newline
Date: 2021/11/25
\newline

Marine mammals like sea lions and seals (e.g., pinnipeds) play important roles in marine ecosystems by regulating the populations of other marine creatures. However, for most of human history, marine mammals have been exploited for their meat, their skins, or to keep them from “stealing” from fisheries or damaging fishing gear.This exploitation brought many marine mammal populations to the brink of extinction in the 18th and 19th centuries.

In the 20th and 21th century, the widespread introduction of protective legislation and policy has allowed many marine mammal populations to recover(e.g. the South American sea lion in Chile). Unfortunately, this recovery has once again fired up conflict between these animals and fisheries. Conflict imposes large negative impacts on fisheries, but also on the recovery prospects of these animal populations. 

Modelling the dynamics of this conflict and quantify its impacts on both fisheries and marine mammal populations can help us to better understand the conflict and give potential help for policy making. After doing Best-Worth Scaling Survey in Chile, the researches found the key driver of conflict between fisheries and marine mammal. Moreover, they did some cluster analysis to cluster different interviewees into four groups. Understanding perceptions of different groups of fishers helps us better target solutions. The researchers are also interested in the future dynamic of the marine mammal population with the influence of culls and extreme climate events. They use population projections and matrix population models to assess the species life cycle. The extreme climate influence is represented by the markov matrix through the year. The result shows that the species will decline over time. And the culls and extreme climate events will increase the speed of it. Although the researchers pointed out the uncertainty in vital rates, we need to focus on this question seriously.


   
\newpage


\section{Georgina Mace Centre for the Living Planet: Making Space for Nature}
Lecturer: John Lawton
\newline
Date: 2022/1/17
\newline

Data on the parlous state of many populations of UK organisms prompted a re-evaluation of conservation policy, for which, the reporters panel title is "make space for nature". The examine evidence on the extent to which England's collection of wildlife sites represents a coherent and resilient ecological network capable of adapting to the challenge of climate change and other pressures. Moreover, the report want to solve the question Whether a more inter-connected network would be more effective today and in the future.

Real protected area in England is about 10\%. They concluded that England's wildlife sites do not comprise a coherent and resilient network. The possible solution had been listed: "1.More new sites, 2.Bigger size of current protecting sites, 3.Better quality of management,4.More joined areas e.g. physical corridors." Interestingly, The bigger reserved region will be more cost-efficient,because the management intensity could be low to maintain the natural ecosystem. The isolation of wild sites will lead to serious species distinct. Therefore, the four possible solutions can help us to better develop the conservation area in both functional and cost-effective way.

This Seminar is about the report, the science underpinning it, the science/policy interface, how we can conserve the wildlife and how important it is.


\newpage






\section{No rapid adaptation to heatwaves in replicated populations with manipulated evolutionary potential}
Lecturer:  Stewart Plaistow, University of Liverpool
\newline
Date: 2022/2/3
\newline

The world is changing at a rapid pace, which is mostly deduced by human. How do organisms rapidly adapt to novel environment. The talk shows two hypothesis. Hypothesis 1: standard genetic variation (e.g. allelic variation and Mendelian inheritance ) . Hypothesis 2: phenotype plasticity -The ability of a genotype to produce distinct phenotypes when exposed to different enviornments throughout its ontogeny
- And environment induced plasticity & non-genetic inheritance (Developmental environment, Epigenetic inheritance) facilitate rapid evolution

To test the hypothesis, They use daphnia spp. as a model system. They found that the plastic responses to each environment are unique. The amount and orientation of genetic variation in each environment is also unique (G-matrices are not stable). high temperatures reduces evolutionary potential and low food reduces evolutionary potential. By mimicking real heatwaves in the mesocosms, they found that the heatwave will lead to short-term transit in phenotype. They use some experiment to exclude the effect of selection. After that, they give out a summary:evolved plastic response in population is sufficient for dealing with heatwaves. an increased frequency of natural heatwaves did not generate rapid adaptation. Exposure to low doses of pollutants causes epigenetic signature (with some overlap). The epigenetic factors may influence the persistent evolutionary dynamics. Moreover, an increased frequency of natural heatwaves did not induce rapid adaptation in D. magna populations as reported in previous studies.


\newpage





\section{Using 30 years of UK monitoring data to identify the impacts of chemicals on invertebrates}
Lecturer: Andrew Johnson
\newline
Date: 2022/2/10
\newline

They use the chemicals threshold and ecotoxicity found in laboratory to test the organisms in the field. They bring different data together, for example, biological data collection (different biological diversity metrics) for macroinvertebrate and fish, physical measurements such as temperature and water flow, chemical measurements including metals. The statistics shows that the improvement of biodiversity happens in both rural and urban area. They test about 50 variables including physical ,chemical,and biological variables. They are trying to see the impact of different variables by various models including GLMM,GAMM and bayesian models. However, not every model says the same thing.The statistical analysis using GLMM and GAMM is not yet complete, but reductions in BOD, PO4 and copper as well as increases in DO are currently noticed as a potential factors that may driven the effect. The land use environment appears to have declined compared to the improved situation of freshwater invertebrates. The study found that the UK does a good job of preserving freshwater biodiversity, but struggles to manage land ecosystems.


\newpage





\section{Genetic and epigenetic variation in mycorrhizal fungi and its role in altering plant growth and carbon storage in tropical soils}
Lecturer: Ian Sanders
\newline
Date: 2022/2/17
\newline


Malnutritions kills more people annually than the "Global killer Diseases" and severely limits. About 1 billion people eat "Cassava" everyday. Compared with other crop like rice, Cassava have more stable yield. The target research species is mycorrhizal fungi, which are found association with the roots, and have the potential to help increase global food production. The goal of the research is to use genetic variation (dikaryote and homo karyon, proportion or allele) of fungi to find potential strategy for global starvation.

The answer is different genetic strains of mycorrhizal fungi can lead to massive growth of yield. Based on this exciting result, they did a control experiment between homokaryon parent and Dikaryon parent to test the source of yield variation. The results shows that plants inoculated with homokaryon siblings vary as much is growth as those inoculated with dikaryons. The next question is how AMF interactions with the plant and soil microbiome. By adding the fungus, the results shows that the diversity of soil community will be altered. Network analysis indicate that the genotype difference of the inoculation leads to the variation of OTU connection. 16S barcoding indicate no siginificant difference in bacterial taxa level. However, metagenome study shows that different AMF strongly affect gene richness and gene diversity of the soil bacterial metagenome.

In conclusion, inoculation of two sibling fungi cause huge changes in yield growth, and other factors:
1.alter soil aggregation and carbon storage in only 1 year. 2.Alter microbial respiration. 3.Increase or decrease gene richness diversity in the bacterial metagenome without obviously changing taxonomic composition. 4. Alter gene composition of the metagenome.






\newpage

\section{Gut microbiome variation and fitness within a natural vertebrate population}
Lecturer: Sarah Worsley
\newline
Date: 2022/3/10
\newline

The researchers wants to test the Gut microbiome variation,fitness and senescence in Seychelles warbler. Gut microbiota can influence many fundamental host processes, including digestion, behaviour and immunity. The diversity of gut microbiome and the number of mutualistic, beneficial bacteria tends to be high in healthy captive animals. However, captive and wild animals are exposed to different pressures.The gut microbiome of wild animals differs across species, populations and individuals. The reason could be Environment factors like sampling season, territory quality, time of day and host factors like species,sex,age stage, genetic heterozygosity (low heterozygosity may lead to high inbreeding which may play a role in shaping the host fitness microbiome), Variation at immune genes.

They did alpha diversity analysis and beta diversity analysis in the gut microbiome. After that, they want to figure out what is the diving factors to the variation. Environmental factors like ages and seasons are considered as the potential driving factors. They found no relationship between shannon index and body mass. They found no relationship between survival and microbiome alpha diversity. 

In conclusion, they found some of the factors that may shape the variation of the microbe. Moreover, They also found possible link to fitness:gut microbiome composition differs between individuals that survived versus those that died. They may use metagenomic sequencing to get the functional profile of the samples in the future study.


\newpage

\section{The limits to ecological limits to diversification}
Lecturer: Rampal
\newline
Date: 2022/3/17
\newline

“As species of the same genus have usually, though by no means invariably, some similarity in habits and constitution, and always in structure, the struggle will generally be more severe between species of the same genus, when they come into competition with each other, than between species of distinct genera”. This statement by Darwin in the Origin of Species (Darwin, 1859), known as the competition-relatedness hypothesis. In today's seminar, researches want to figure out that if this feedback only occur between close relatives, or do distant relatives also influence their diversification. Among all the experiment niches, for the sake of that their boundedness facilitates an overview of all potential competitors, islands form ideal systems could be used to answer this question. They use statistical model and computational power mainly based on aDAISIE (Dynamic Assembly of Island biota through Speciation Immigration and Extinction) framework and Bootstrap analysis to figure out the question and overcome the current bias and unsolved questions. The result shows that clade-specific (CS) diversity-dependence model fits better than island-wide (IW) diversity-dependence. Moreover, by using phylogenetic data rather than distributional data , they proved that this method does allow for inferring the phylogenetic limits to ecological limits to diversification. For example, based on the model species they used :Hispaniolan Eleutherodactylus frogs, they find a negative feedback of diversity on colonization and speciation rates, suggesting that ecological limits play an important role. 

\newpage

\section{Microbial processes impacting soil carbon cycling under anthropogenic change: trait-based scaling from individuals to the ecosystem}
Lecturer: Ashish A Malik
\newline
Date: 2022/5/12
\newline

Microorganisms play a critical role in terrestrial carbon cycling by controlling the fate of recent plant carbon inputs as well as protected soil organic carbon. Soil carbon stocks can reflect many important indicators, such as a balance between microbial decomposition of organic carbon and stabilisation of microbial assimilated carbon. The Carbon-microorganisms balance can shift under altered environmental conditions.Therefore, understanding microbial physiology and how the balance shift can help predict soil carbon changes and improve climate change feedback predictions. Plant ecology was able to respond to the need for climate science research, whereas microbial ecology obviously was lagging because invisibility and culture difficulty make microbes hard to study. So although sequencing technologies revolutionized microbial ecology in the way we study microbial taxonomy function across space and time. In the seminar, the lecturer tried to looking at a trait based (YAS) framework so that they can start linking microbial processes to ecosystem scale processes. They use 2 different examples to explain the influence of environmental conditions changes the microbial community. Consequently, they found different strategies for different situation including a high yield strategy, but also found an acquisition strategy that was geared towards uptake of simple or easily available compounds. This seminar shows that the trait-based framework did is promising in climate change study.



\newpage

\section{Inference of functional consequences of biodiversity loss from traits}
Lecturer:Matthias Schleuning
\newline
Date: 2022/5/19
\newline

In most groups, 25\% of species are threatened with extinction. The rapid pace of global climate change is causing biodiversity loss at an unprecedented rate. The biodiversity crisis can lead to economic losses. For example, a decrease in insects can lead to a decrease in crop production. The interactions between animals and plants make it difficult to understand the functional consequences of biodiversity. The researchers are addressing this challenge and are developing broadly applicable trait concepts. For example, Adult mass is a key factor in determining how well the creature will tolerate climate change. Moreover, Functional traits can be used to characterize and infer species interactions.The lecture showed an example in which the extinction of birds can cause a non-linear change in the function that predicts how plants will spread their seeds. This effect is likely to be more pronounced with increasing temperatures. Overall, these studies highlight the big potential of trait-based research for unravelling the complexity and understanding the functioning of real-world ecosystems.
\newpage


\section{Climate change impacts on the physical environment of lakes }
Lecturer: Iestyn Woolway
\newline
Date: 2022/6/16
\newline

The theme of the seminar is to see how the lake respond to the long term environmental changes. They use both lake tahoe and satellites to monitor the changes of lake. From the data collected from those system, they simulating lake response to environmental change. A thermocline is the transition layer between the warmer mixed water at the surface and the cooler deep water below. 
Based on the seasonal difference of thermocline, they coined several different descriptive term.
By looking at the ice cover in a longnitude data, the freeze date become more and more late in recent years, while the thaw date become earlier. Moreover, they demonstrated that the safe ice thickness is declining significantly which will influence the culture and transportation. Stratification of the lake may be more earlier because of the climate change. In the large global data set, they find that there is about 81\% lake have a warming pattern. In conclusion,
climatic warming can have a dramatic influence on the physical environment of lakes, including a decline in lake ice cover, thermal stratification, and increasing evaporation rates, with knock-on effects on lake water level and extent.

\newpage

\end{document}
