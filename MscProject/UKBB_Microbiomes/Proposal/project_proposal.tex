\documentclass[11pt,a4paper]{article}
\usepackage[utf8]{inputenc}
\usepackage{setspace}
\usepackage{graphicx}
\usepackage{float}
\usepackage{subfigure}
\usepackage[justification=centering]{caption}
\usepackage{amsmath}

\usepackage{natbib}
\bibliographystyle{abbrvnat}
\setcitestyle{authoryear} %Citation-related commands

\renewcommand{\baselinestretch}{1.5}

\title{   \textbf{ Dissertation Proposal } \thanks{MSc project}  \\
Using 16S Gut Microbe Sequencing to Explore the Association Between Microbiome Composition with Different Health Metrics}


\author{ \textbf{Author:} Congjia Chen \\ Congjia.Chen21@imperial.ac.uk \\ \\ \textbf{Supervisor:} Dr. Peter Graystock \\ p.graystock@imperial.ac.uk \\ \\  }


\date{ \\ \\March 2022 \\ Department of Life Sciences, Imperial College of Science, Technology and Medicine}


\begin{document}
\maketitle

\begin{center}
\par\textbf{ Keywords  } \\
Bee;Microbiome;Taxonomy Composition;RAD Sequencing;16S sequencing; Parasite-host interaction 

\end{center}

\newpage

\noindent\textbf{1.Introduction}\\
Bumble bees (Bombus) are vitally important pollinators of wild plants and agricultural crops worldwide\citep{kwong_gut_2016}. However, Evidence have suggested population declines in bumble bees species in Britain\citep{powney_widespread_2019}. Recent bee population declines have drawn attention to potential agents affecting their health, including their microbiota. Indeed, there is evidence that the bee gut bacteria help defend against pathogen infection, in ways that are still poorly understood\citep{kwong_gut_2016,koch_socially_2011,koch_gut_2012}\\
In recent years, microbiome studies has increased dramatically, boosted by advances of sequencing technology\citep{knight_best_2018}. Such studies has revealed the wealth of the data, which has yielded tremendous insight into the nature of the microbial communities\citep{cullen_emerging_2020}. There are evidence that microorganisms within animal guts can benefit their hosts,such as helping to digest food, providing nutrients and even protecting against invasion by pathogens and parasites\citep{cullen_emerging_2020}. However, while the importance of gut microbial communities is increasingly appreciated, the properties and mechanisms that control these communities are far from understanding. \\
This project will focus on the bumble bees sampled in UK. The main idea of this project is to explore the health of our pollinators linking microbial composition with health metrics (e.g. presence of pathogen DNA and beneficial bacteria in the bees).16S amplicon gut microbe sequencing will be applied to get the microbial composition and information from over 200 UK bumble bee samples. RAD-sequencing will be used to give insight of the population genetics (inbreeding, diversity etc)  of bumble bees\citep{andrews_harnessing_2016}.\\
\textbf{2.Proposed methods}\\
\textbf{Basic Microbial Composition Analysis:}
 The main procedure of the amplicon sequencing data analysis includes(1) Do demultiplexing and quality control of sequencing data (2) Denoise reads into operational taxonomic units (OTUs) (3) Taxonomic classification (4) Subsequent advanced data analysis (e.g. Alpha and beta diversity analyses) \\
\textbf{Host Population Genetics:}
Using RAD-Sequencing data to explore the population genetics aspect of humble bees. From the RAD sequencing data I will explore if more intracellular parasites like Wolbachia and Sodalis can be detected by aligning the reads against the endosymbiont genome.\\
\textbf{Advanced Data Analysis:} The main idea of this project is to identify the association between microbial composition and the host health conditions. I will use several metrics to represent the health conditions of bees. One of the metrics will be the parasite loads which is based on visual observations of transmittable spores/cells of the parasites in the guts of the hosts. Another metrics that we will use is the intracellular parasites occurrence from RAD-sequencing data and 16S sequencing data. Additionally, if time permit, I will try to explore the link between the host population genetics and microbiome composition\\
\textbf{3.Anticipated outputs and outcomes}
Anticipated outputs is to identify the association between microbial compositions and health conditions incorporating different population status, such as species, sampling sites\\
\textbf{4.Project feasibility}
The Project is feasible as the plan shown below:

\begin{figure}[H]
    \centering
    \includegraphics[width=\textwidth]{./Gantt_chart.pdf}
    \caption{Gantt chart}
\end{figure}

\noindent\textbf{5.An itemized budget}
8Tb External hard-drive to backup data (£200)


\newpage
  
\bibliographystyle{unsrtnat}
\bibliography{References.bib}

\newpage

\noindent\textbf{Supervisor Statement} \\


\noindent\textbf{I have seen and approved the proposal and the budget.} \\
\\
Supervisor name: Peter Graystock\\
Signature: \\




\noindent Date: 4/4/2022\\

\end{document}
